% !TEX TS-program = xelatex
% !TEX encoding = UTF-8 Unicode

% GSET Summer 2021 - Tennessee Technological University
% Tristan Hill - June 11, 2021
% Turtorial 5 - Your Name in Ascii

\documentclass[12pt]{article}

% Custom Preamble
\usepackage{/home/thill/Documents/lectures/cpp_workshop/modules/cpp_tutorial} 

% Title and Misc
\newcommand{\MNUM}{4} %Module Number
\newcommand{\MNAME}{Introduction to C++} %Module Name
\newcommand{\TNAME}{Your Name in ASCII} %Tutorial Name
\pagestyle{myheadings}
\markright{{\large GSET - Programming with Mr. Hill}}

\begin{document}

\thispagestyle{plain}

\begin{center}
   {\bf \large GSET - Programming with Mr. Hill - Summer 2021} \vspace{5mm}\\
   {\bf \Large \MNAME \hspc -  Tutorial\hspc\MNUM\hspc - \TNAME}\vspace{3mm}\\
   
\end{center}

 %\hspace*{3cm}\includegraphics[scale=.15]{quad_equ.png} 

\begin{description}[labelindent=1cm]
	
	\item[\textbf{\underline{Overview:}}] \hfill \vspace{3mm}\\
	We are going to write a C++ program to display your name in ASCII characters on the screen. 
	
	\item[\textbf{\underline{System Requirements:}}] \hfill \vspace{0mm}

\begin{itemize}
	\item {\bf Computer}: A computer is required to complete this tutorial. Any OS should work.
	\item {\bf C++:} You can use the online C++ compiler (\href{https://www.onlinegdb.com/online\_c++\_compiler}{OnlineGDB} ) or a C++ compiler of your choice.
\end{itemize}

	\item[\textbf{\underline{Problem Statement:}}] \hfill \vspace{0mm}
	
	\begin{itemize}

		\item Given: Your name in character format.
		
		\item Find: Your name in ASCII decimal and hex codes.
		 
	\end{itemize}

\item[\textbf{\underline{Program Minimum Requirements:}}] \hfill \vspace{0mm}

The program should accomplish the following tasks. 


\begin{itemize}

	\item Your name should be stored as an array of characters.

	\item Your program should print your name to the screen in character format.
	
	\item Your program should print your name to the screen in ASCII decimal code format.
	
	\item Your program should print your name to the screen in ASCII hex code format.

\end{itemize}	 
	Optional Advanced Features:
\begin{itemize}
	\item Your name should be read from the user during program execution
		
	\item The program should also include your last name and possibly other info.
    
    \item Your program should make a cool beeping sound when is it complete.
    
    \item Your program should make return the length (number or characters) in your name.
    
\end{itemize}	
\newpage

\item[\textbf{\underline{Example Code:}}] \hfill \vspace{0mm}
\begin{enumerate}
    \item This is the C style way to output text.
	%\begin{minted}{cpp}
	\begin{lstlisting}

// Arrays of Characters - GSET - Summer 2021 
	
#include <iostream>

using namespace std;

int main()
{

	char myname[]={"Tristan"};
	
	char c = 'T';
	
	cout<<"Hello World\n";
	
	cout<<myname<<endl; 
	
	cout<<myname[1]<<endl;
	
	cout<<(int)c<<"\a" <<endl; 
	
	return 0;
}


	\end{lstlisting}
	%\end{minted}
		
\end{enumerate}

	\item[\textbf{\underline{Part 3 - Testing:}}] \hfill \vspace{0mm}
	\begin{enumerate}
	
		\item Complete the C++ code to the solve the problem described. \\\\
		
		\item Test your code with different inputs. Is the answer correct? How do you know? Are there certain inputs that do not work? \\\\
		
	
		\item Save your code with the download button or use copy and paste. You can view and edit the code in any text editor. Also, save a copy of the program output for your tutorial summary. \\\\

	\end{enumerate}

\newpage
\item[\textbf{\underline{Solution Code:}}] \hfill \vspace{0mm}
%
%\begin{lstlisting}
%
%\end{lstlisting}
%
%\item[\textbf{\underline{Tutorial Complete:}}] \hfill \vspace{3mm}\\ 
%	Congratulations, after completing {\it Tutorial 2 - Quadratic Equation}, you have begun learning to program in C++! You are now ready to start learning about more complex data structures and program flow. \\
%

\newpage
\item[\textbf{\underline{Tutorial Summary:}}] \hfill \vspace{3mm}\\ 
Write a brief summary of what you accomplished and what you struggled with the most. 

Include the following items in the summary:
\begin{itemize}

\item a copy of the output of your program
\item a description of what the program does and how to use it

\end{itemize}


%\item[\textbf{\underline{Submission on Teams:}}] \hfill \vspace{3mm}\\ 
%Use the appropriate shared folder on Teams to submit your program and summary. Submit the fol1owing items with your TNTech username in the filenames as shown below. \vspace{0mm}\\
%
%\underline{Files for Tutorial 1 (TNTech Username : twhill21)}
%
%\begin{itemize}
%
%\item Tutorial Summary: \textbf{ twhill21\_summary2.txt}
%
%\item C++ Source Code: \textbf{ twhill21\_tutorial2.cpp}
%
%\end{itemize}


\end{description}
\end{document}

