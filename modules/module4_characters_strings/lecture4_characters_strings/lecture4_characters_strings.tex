% !TEX TS-program = xelatex
% !TEX encoding = UTF-8 Unicode

% Tennessee Technological University
% ENGR1120-021 - GSET - Summer 2021
% Tristan Hill - June 11, 2021
% Module 4 - Characters and Strings
% Lecture 4 

\documentclass[fleqn]{beamer} % for presentation (has nav buttons at bottom)

\usepackage{/home/thill/Documents/lectures/cpp_workshop/modules/cpp_lectures}

\newcommand{\MNUM}{4\hspace{2mm}} % Module number
\newcommand{\TNUM}{---\hspace{2mm}} % Topic number - single topic for now
\newcommand{\moduletitle}{Characters and Strings} % Titles and Stuff
%\newcommand{\topictitle}{---} 

\newcommand{\sectiontitleI}{Others types of Data} % More Titles and Stuff
\newcommand{\sectiontitleII}{The char and small integer type}
\newcommand{\sectiontitleIII}{The ASCII Table}
\newcommand{\sectiontitleIV}{C++ Example}
\newcommand{\sectiontitleV}{Tutorial 4 - asdf }

\newcommand{\btVFill}{\vskip0pt plus 1filll}

\setbeamercolor{title in head/foot}{fg=TTUgold} % this needs work...

\title{GSET - Programming with Mr. Hill}
\author{Tristan Hill\vspc \hspc Tennessee Technological University \hspc}
\date{Summer 2021}

\begin{document}

\lstset{language=MATLAB,basicstyle=\ttfamily\small,showstringspaces=false}

\frame{\titlepage \center\begin{framed}\Large \textbf{Module \MNUM - \moduletitle}\end{framed} \vspace{5mm}}


% Section 0 - Outline
\frame{
	
	\large \textbf{Module \MNUM - \moduletitle} \vspace{3mm}\\
	
	\begin{itemize}
	
		\item \hyperlink{sectionI}{\sectiontitleI} \vspc % Section I
		\item \hyperlink{sectionII}{\sectiontitleII} \vspc % Section II
		\item \hyperlink{sectionIII}{\sectiontitleIII} \vspc %Section III
		\item \hyperlink{sectionIV}{\sectiontitleIV} \vspc %Section IV	
		\item \hyperlink{sectionV}{\sectiontitleV} \vspc %Section V
	
	\end{itemize}

}


% Section I
\section{\sectiontitleI}

	% Section I - Frame I
	\begin{frame}[label=sectionI,containsverbatim] \small
		\frametitle{\sectiontitleI}
		We have used {\PN scalar} variables in the previous exercises. Have you used this term before?
	
	\end{frame}

	% Section I - Frame II
	\begin{frame} \small
		\frametitle{\sectiontitleI}
	
			Now, we are going to learn to use array variables in C++. \vspace{5mm}\\
			
			\begin{itemize}
			
			\item What is a array? \vspace{5mm} \\
			
			\item Why are arrays important in computer programming? \vspace{5mm}\\
			
				\begin{itemize}
				
					\item 
					
					\item 
					
					\item 
				
				\end{itemize}
	
			\end{itemize}
	
		\btVFill
		\tiny{some reference}	
	\end{frame}


% Section II
\section{\sectiontitleII}

	% Section II - Frame I
	\begin{frame}[label=sectionII,containsverbatim] \small
	\frametitle{\sectiontitleII}
	
	Variables are stored in the \href{https://en.wikipedia.org/wiki/Random-access_memory}{RAM (Random Access Memory)}. Each variable has a {\BL name}, a {\GR value}, and a {\PR memory address}.  \vspace{5mm}\\
	
	\begin{multicols}{2}
		
		\begin{lstlisting}
		
		...
		
		int x;
		float our_value=43.5;
		
		
		
		...			
		
		\end{lstlisting}
		
		\renewcommand*{\arraystretch}{1.5}
		\begin{tabular}{c|c|c} 
			Name&Value&Memory Address\\ \hline
			& & \\ \hline
			& & \\ \hline
			& & \\ \hline
			& & \\ \hline
			& & \\ \hline
			& & \\ \hline
		\end{tabular}
		
			
		
	\end{multicols}
	
	\btVFill
	
	%\tiny{reference: \href{https://www.cplusplus.com/doc/tutorial/operators/}{cplusplus.com} } 	
	
\end{frame}

	% Section II - Frame II
	\begin{frame} \small
		\frametitle{\sectiontitleII}
		
		
		\btVFill
		%\tiny{ref: \href{some link}{some text}}
	\end{frame}	


% Section III
\section{\sectiontitleIII}

	% Section III - Frame I
	\begin{frame}[label=sectionIII] \small
		\frametitle{\sectiontitleIII}
	
		
	\end{frame}


% Section IV
\section{\sectiontitleIV}	
	% Section IV - Frame I
	\begin{frame}[label=sectionIV] \small
		\frametitle{\sectiontitleIV}    
	

		\btVFill
		%\tiny{ref: \href{some link}{some text}} 
	\end{frame}

% Section V
\section{\sectiontitleV}	
	% Section V - Frame I
	\begin{frame}[label=sectionV,containsverbatim] \small
	\frametitle{\sectiontitleV}    
	

	\end{frame}


\end{document}

