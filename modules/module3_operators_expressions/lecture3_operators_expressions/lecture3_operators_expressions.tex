% !TEX TS-program = xelatex
% !TEX encoding = UTF-8 Unicode

% Tennessee Technological University
% ENGR1120-021 - GSET - Summer 2021
% Tristan Hill - June 08, 2021
% Module 3 - Operators and Expressions
% Lecture 3 


\documentclass[fleqn]{beamer} % for presentation (has nav buttons at bottom)

\usepackage{/home/thill/Documents/lectures/cpp_workshop/modules/cpp_lectures}

\newcommand{\MNUM}{2\hspace{2mm}} % Module number
\newcommand{\TNUM}{---\hspace{2mm}} % Topic number - single topic for now
\newcommand{\moduletitle}{Operators and Expressions} % Titles and Stuff
%\newcommand{\topictitle}{---} 

\newcommand{\sectiontitleI}{Operators in C++} % More Titles and Stuff
\newcommand{\sectiontitleII}{Arithmetic Operators}
\newcommand{\sectiontitleIII}{Operator Precedence}
\newcommand{\sectiontitleIV}{What is an Expression?}
\newcommand{\sectiontitleV}{Tutorial 2 - Quadratic Equation - Solution}

\newcommand{\btVFill}{\vskip0pt plus 1filll}

\setbeamercolor{title in head/foot}{fg=TTUgold} % this needs work...

\title{GSET - Programming with Mr. Hill}
\author{Tristan Hill\vspc \hspc Tennessee Technological University \hspc}
\date{Summer 2021}

\begin{document}

\lstset{language=MATLAB,basicstyle=\ttfamily\small,showstringspaces=false}

\frame{\titlepage \center\begin{framed}\Large \textbf{Module \MNUM - \moduletitle}\end{framed} \vspace{5mm}}


% Section 0 - Outline
\frame{
	
	\large \textbf{Module \MNUM - \moduletitle} \vspace{3mm}\\
	
	\begin{itemize}
	
		\item \hyperlink{sectionI}{\sectiontitleI} \vspc % Section I
		\item \hyperlink{sectionII}{\sectiontitleII} \vspc % Section II
		\item \hyperlink{sectionIII}{\sectiontitleIII} \vspc %Section III
		\item \hyperlink{sectionIV}{\sectiontitleIV} \vspc %Section IV	
		\item \hyperlink{sectionV}{\sectiontitleV} \vspc %Section V
	
	\end{itemize}

}


% Section I
\section{\sectiontitleI}

	% Section I - Frame I
	\begin{frame}[label=sectionI] \small
		\frametitle{\sectiontitleI}
		
		There are {\it many different} {\BL operators} used in computer programming.  \vspace{5mm}\\
			
		\renewcommand*{\arraystretch}{1.5}
		\begin{tabular}{c|c|c} 
			Category&Operators&Example\\ \hline
			Assignment&$=$& \\ \hline
			Arithmetic&$+ \hspc - \hspc * \hspc / \hspc \% \hspcc {\RD \hat{} }$& \\ \hline
			Relational&$== \hspc !\hspace{-1mm}= \hspc > \hspc < \hspc >= \hspc <= $& \\ \hline
			Logical&$! \hspc \&\& \hspc || $& \\ \hline
			Bitwise&$\& \hspc | \hspc \hat{} \hspc \sim \hspc << \hspc >>$ & \\ \hline
		\end{tabular}
	
		\vspace*{10mm}{\tiny See the link below for a complete list of C++ operators.}
		
		\btVFill
		
		\tiny{reference: \href{https://www.cplusplus.com/doc/tutorial/operators/}{cplusplus.com} } 	
			
	\end{frame}

	% Section I - Frame II
	\begin{frame} \small
		\frametitle{\sectiontitleI}
	
			There are more. What else could there be?
	
		\btVFill
		\tiny{some reference}	
	\end{frame}


% Section II
\section{\sectiontitleII}

	% Section II - Frame I
	\begin{frame}[label=sectionII] \small
		\frametitle{\sectiontitleII}
		
		\renewcommand*{\arraystretch}{1.5}
		\begin{tabular}{c|c|c}
			
			Operator& Description & Example \\ \hline
			$+$&Addition& \\ \hline
			$-$&Subtraction& \\ \hline
			$*$&Multiplication& \\ \hline
			$/$&Division& \\ \hline
			$\%$&Modulo& \\ \hline
	
		\end{tabular}
	
		\end{frame}

	% Section II - Frame II
	\begin{frame} \small
		\frametitle{\sectiontitleII}
		
		\vspace{5MM}
		
		Which one of these is not like the other?
		
		\vspace{10mm}
		
		{\Large 	$= \hspace{12mm} + \hspace{12mm} - \hspace{12mm} * \hspace{12mm} / \hspace{12mm} \% \hspace{12mm} {\RD \hat{} }$ }
		
		\vspace{10mm}
		
		Why?
		
		\btVFill
		\tiny{ref: \href{some link}{some text}}
	\end{frame}	


% Section III
\section{\sectiontitleIII}

	% Section III - Frame I
	\begin{frame}[label=sectionIII] \small
		\frametitle{\sectiontitleIII}
		
		\includegraphics[scale=.19]{operator_precedence.png}
		
		
		\btVFill
		\tiny{reference: \href{https://www.cplusplus.com/doc/tutorial/operators/}{cplusplus.com} } 
	\end{frame}

	% Section III - Frame II
	\begin{frame} \small
		\frametitle{\sectiontitleIII}
		
		\underline{Question:} What was the concept of operator precedence called in mathematics class? \vspace{15mm}
		
		\underline{Answer:}
		
	\end{frame}


% Section IV
\section{\sectiontitleIV}	
	% Section IV - Frame I
	\begin{frame}[label=sectionIV] \small
		\frametitle{\sectiontitleIV}    
	
		\vspace{15mm}
	
		{\it An expression is a sequence of operators and their operands, that specifies a computation.}

		\btVFill
		\tiny{reference: \href{https://www.cplusplus.com/doc/tutorial/operators/}{cplusplus.com} } 
	\end{frame}

% Section V
\section{\sectiontitleV}	
	% Section V - Frame I
	\begin{frame}[label=sectionV,containsverbatim] \small
	\frametitle{\sectiontitleV}    
	

	\end{frame}


\end{document}

